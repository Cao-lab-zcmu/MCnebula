\documentclass[]{tufte-handout}

% ams
\usepackage{amssymb,amsmath}

\usepackage{ifxetex,ifluatex}
\usepackage{fixltx2e} % provides \textsubscript
\ifnum 0\ifxetex 1\fi\ifluatex 1\fi=0 % if pdftex
  \usepackage[T1]{fontenc}
  \usepackage[utf8]{inputenc}
\else % if luatex or xelatex
  \makeatletter
  \@ifpackageloaded{fontspec}{}{\usepackage{fontspec}}
  \makeatother
  \defaultfontfeatures{Ligatures=TeX,Scale=MatchLowercase}
  \makeatletter
  \@ifpackageloaded{soul}{
     \renewcommand\allcapsspacing[1]{{\addfontfeature{LetterSpace=15}#1}}
     \renewcommand\smallcapsspacing[1]{{\addfontfeature{LetterSpace=10}#1}}
   }{}
  \makeatother

\fi

% graphix
\usepackage{graphicx}
\setkeys{Gin}{width=\linewidth,totalheight=\textheight,keepaspectratio}

% booktabs
\usepackage{booktabs}

% url
\usepackage{url}

% hyperref
\usepackage{hyperref}

% units.
\usepackage{units}


\setcounter{secnumdepth}{-1}

% citations


% pandoc syntax highlighting
\usepackage{color}
\usepackage{fancyvrb}
\newcommand{\VerbBar}{|}
\newcommand{\VERB}{\Verb[commandchars=\\\{\}]}
\DefineVerbatimEnvironment{Highlighting}{Verbatim}{commandchars=\\\{\}}
% Add ',fontsize=\small' for more characters per line
\newenvironment{Shaded}{}{}
\newcommand{\AlertTok}[1]{\textcolor[rgb]{1.00,0.00,0.00}{\textbf{#1}}}
\newcommand{\AnnotationTok}[1]{\textcolor[rgb]{0.38,0.63,0.69}{\textbf{\textit{#1}}}}
\newcommand{\AttributeTok}[1]{\textcolor[rgb]{0.49,0.56,0.16}{#1}}
\newcommand{\BaseNTok}[1]{\textcolor[rgb]{0.25,0.63,0.44}{#1}}
\newcommand{\BuiltInTok}[1]{#1}
\newcommand{\CharTok}[1]{\textcolor[rgb]{0.25,0.44,0.63}{#1}}
\newcommand{\CommentTok}[1]{\textcolor[rgb]{0.38,0.63,0.69}{\textit{#1}}}
\newcommand{\CommentVarTok}[1]{\textcolor[rgb]{0.38,0.63,0.69}{\textbf{\textit{#1}}}}
\newcommand{\ConstantTok}[1]{\textcolor[rgb]{0.53,0.00,0.00}{#1}}
\newcommand{\ControlFlowTok}[1]{\textcolor[rgb]{0.00,0.44,0.13}{\textbf{#1}}}
\newcommand{\DataTypeTok}[1]{\textcolor[rgb]{0.56,0.13,0.00}{#1}}
\newcommand{\DecValTok}[1]{\textcolor[rgb]{0.25,0.63,0.44}{#1}}
\newcommand{\DocumentationTok}[1]{\textcolor[rgb]{0.73,0.13,0.13}{\textit{#1}}}
\newcommand{\ErrorTok}[1]{\textcolor[rgb]{1.00,0.00,0.00}{\textbf{#1}}}
\newcommand{\ExtensionTok}[1]{#1}
\newcommand{\FloatTok}[1]{\textcolor[rgb]{0.25,0.63,0.44}{#1}}
\newcommand{\FunctionTok}[1]{\textcolor[rgb]{0.02,0.16,0.49}{#1}}
\newcommand{\ImportTok}[1]{#1}
\newcommand{\InformationTok}[1]{\textcolor[rgb]{0.38,0.63,0.69}{\textbf{\textit{#1}}}}
\newcommand{\KeywordTok}[1]{\textcolor[rgb]{0.00,0.44,0.13}{\textbf{#1}}}
\newcommand{\NormalTok}[1]{#1}
\newcommand{\OperatorTok}[1]{\textcolor[rgb]{0.40,0.40,0.40}{#1}}
\newcommand{\OtherTok}[1]{\textcolor[rgb]{0.00,0.44,0.13}{#1}}
\newcommand{\PreprocessorTok}[1]{\textcolor[rgb]{0.74,0.48,0.00}{#1}}
\newcommand{\RegionMarkerTok}[1]{#1}
\newcommand{\SpecialCharTok}[1]{\textcolor[rgb]{0.25,0.44,0.63}{#1}}
\newcommand{\SpecialStringTok}[1]{\textcolor[rgb]{0.73,0.40,0.53}{#1}}
\newcommand{\StringTok}[1]{\textcolor[rgb]{0.25,0.44,0.63}{#1}}
\newcommand{\VariableTok}[1]{\textcolor[rgb]{0.10,0.09,0.49}{#1}}
\newcommand{\VerbatimStringTok}[1]{\textcolor[rgb]{0.25,0.44,0.63}{#1}}
\newcommand{\WarningTok}[1]{\textcolor[rgb]{0.38,0.63,0.69}{\textbf{\textit{#1}}}}

% table with pandoc

% multiplecol
\usepackage{multicol}

% strikeout
\usepackage[normalem]{ulem}

% morefloats
\usepackage{morefloats}


% tightlist macro required by pandoc >= 1.14
\providecommand{\tightlist}{%
  \setlength{\itemsep}{0pt}\setlength{\parskip}{0pt}}

% title / author / date
\title{MCnebula workflow for LC-MS/MS dataset analysis}
\author{Lichuang Huang}
\date{}


\begin{document}

\maketitle




\hypertarget{introduction}{%
\subsection{Introduction}\label{introduction}}

This vignette descrip a classified visualization method, called
MCnebula, for the analysis of untargeted LC-MS/MS datasets. MCnebula
utilizes the state-of-the-art computer prediction technology, SIRIUS
workflow (SIRIUS, ZODIAC, CSI:fingerID, CANOPUS), for compound formula
prediction, structure retrieve and classification prediction. MCnebula
integrates an abundance-based class selection algorithm into compound
annotation. The benefits of molecular networking, i.e.~intuitive
visualization and a large amount of integratable information, were
incorporated into MCnebula visualization. With MCnebula, we can switch
from untargeted to targeted analysis, focusing precisely on the compound
or chemical class of interest to the researcher.

\hypertarget{r-and-other-softs-setup}{%
\subsection{R and other softs Setup}\label{r-and-other-softs-setup}}

\hypertarget{r-setup}{%
\subsubsection{R setup}\label{r-setup}}

\begin{Shaded}
\begin{Highlighting}[]
\KeywordTok{library}\NormalTok{(MCnebula)}
\KeywordTok{library}\NormalTok{(dplyr)}
\KeywordTok{library}\NormalTok{(ggplot2)}
\KeywordTok{library}\NormalTok{(ggraph)}
\KeywordTok{library}\NormalTok{(grid)}
\end{Highlighting}
\end{Shaded}

\hypertarget{others-setup}{%
\subsubsection{Others setup}\label{others-setup}}

The prerequisite soft for MCnebula is SIRIUS 4. Users can download that
from \url{https://bio.informatik.uni-jena.de/software/sirius/}.

   In addition, users are encouraged to perform MZmine 2
(\url{https://github.com/mzmine/mzmine2}) for LC-MS/MS data processing.

   Before mass spectrometry data been processed, the raw data should be
converted to .mzML or .mzXML file in most cases. ProteoWizard
(\url{https://proteowizard.sourceforge.io/}) could be implemented for
conversion.

\hypertarget{data-preprocessing}{%
\subsection{Data preprocessing}\label{data-preprocessing}}

\hypertarget{raw-data-processing}{%
\subsubsection{Raw data processing}\label{raw-data-processing}}

For MZmine2 processing, an XML batch file outlined the example
parameters for waters Qtof could be find in
\url{https://github.com/Cao-lab-zcmu/research-supplementary}.

\hypertarget{sirius-computation-workflow}{%
\subsubsection{SIRIUS computation
workflow}\label{sirius-computation-workflow}}

Here we prepared some example files for this vignette to better
illustrate MCnebula workflow.

\begin{Shaded}
\begin{Highlighting}[]
\NormalTok{eg.path \textless{}{-}}\StringTok{ }\KeywordTok{system.file}\NormalTok{(}\StringTok{"extdata"}\NormalTok{, }\StringTok{"raw\_instance.tar.gz"}\NormalTok{, }\DataTypeTok{package =} \StringTok{"MCnebula"}\NormalTok{)}
\NormalTok{tmp \textless{}{-}}\StringTok{ }\KeywordTok{tempdir}\NormalTok{()}
\NormalTok{utils}\OperatorTok{::}\KeywordTok{untar}\NormalTok{(eg.path, }\DataTypeTok{exdir =}\NormalTok{ tmp)}
\NormalTok{mgf.path \textless{}{-}}\StringTok{ }\KeywordTok{paste0}\NormalTok{(tmp, }\StringTok{"/"}\NormalTok{, }\StringTok{"instance5.mgf"}\NormalTok{)}
\CommentTok{\#\#\# show details of .mgf}
\NormalTok{data.table}\OperatorTok{::}\KeywordTok{fread}\NormalTok{(mgf.path, }\DataTypeTok{header =}\NormalTok{ F, }\DataTypeTok{sep =} \OtherTok{NULL}\NormalTok{)}
\CommentTok{\#\textgreater{}                           V1}
\CommentTok{\#\textgreater{}   1:              }\RegionMarkerTok{BEGIN}\CommentTok{ IONS}
\CommentTok{\#\textgreater{}   2:     FEATURE\_ID=gnps1234}
\CommentTok{\#\textgreater{}   3: PEPMASS=468.29557911617}
\CommentTok{\#\textgreater{}   4:               CHARGE=+1}
\CommentTok{\#\textgreater{}   5:               MSLEVEL=1}
\CommentTok{\#\textgreater{}  {-}{-}{-}                        }
\CommentTok{\#\textgreater{} 532:            381.78391 30}
\CommentTok{\#\textgreater{} 533:           382.14539 260}
\CommentTok{\#\textgreater{} 534:            383.14563 10}
\CommentTok{\#\textgreater{} 535:                }\RegionMarkerTok{END}\CommentTok{ IONS}
\CommentTok{\#\textgreater{} 536:}
\end{Highlighting}
\end{Shaded}

If SIRIUS soft has been download locally and the Environment Path has
been set down (for example, set
\texttt{PATH=\$PATH:/you/your\_dir/sirius-gui/bin} in
\texttt{\textasciitilde{}/.bashrc}), the following is availlable:

\begin{Shaded}
\begin{Highlighting}[]
\KeywordTok{system}\NormalTok{(}
  \KeywordTok{paste0}\NormalTok{(}
    \StringTok{"sirius {-}i "}\NormalTok{,}
\NormalTok{    mgf.path,}
    \StringTok{" {-}o test {-}{-}maxmz 800 formula {-}c 50 zodiac structure canopus"}
\NormalTok{    )}
\NormalTok{)}
\end{Highlighting}
\end{Shaded}

A more simpler approach is to use SIRIUS GUI version.

\hypertarget{mcnebula-processing}{%
\subsubsection{MCnebula processing}\label{mcnebula-processing}}

In deed, the computation of SIRIUS is time-consumed. It may cost several
hours even days. Here, we prepared a fairly small dataset (SIRIUS
project space) which has been done from SIRIUS computation. The
following show the detail of this dataset.

\begin{Shaded}
\begin{Highlighting}[]
\KeywordTok{list.files}\NormalTok{(tmp) }\OperatorTok{\%\textgreater{}\%}\StringTok{ }
\StringTok{  }\NormalTok{.[}\KeywordTok{which}\NormalTok{(. }\OperatorTok{!=}\StringTok{ "instance5.mgf"}\NormalTok{)]}
\CommentTok{\#\textgreater{}  [1] "0\_instance5\_gnps1234"                                       }
\CommentTok{\#\textgreater{}  [2] "1\_instance5\_gnps1537"                                       }
\CommentTok{\#\textgreater{}  [3] "2\_instance5\_gnps1539"                                       }
\CommentTok{\#\textgreater{}  [4] "3\_instance5\_gnps1567"                                       }
\CommentTok{\#\textgreater{}  [5] "4\_instance5\_gnps1588"                                       }
\CommentTok{\#\textgreater{}  [6] "canopus\_summary\_adducts.tsv"                                }
\CommentTok{\#\textgreater{}  [7] "canopus\_summary.tsv"                                        }
\CommentTok{\#\textgreater{}  [8] "canopus.tsv"                                                }
\CommentTok{\#\textgreater{}  [9] "compound\_identifications\_adducts.tsv"                       }
\CommentTok{\#\textgreater{} [10] "compound\_identifications.tsv"                               }
\CommentTok{\#\textgreater{} [11] "downloaded\_packages"                                        }
\CommentTok{\#\textgreater{} [12] "formula\_identifications\_adducts.tsv"                        }
\CommentTok{\#\textgreater{} [13] "formula\_identifications.tsv"                                }
\CommentTok{\#\textgreater{} [14] "libloc\_169\_8f48fc7d1db1bb3e.rds"                            }
\CommentTok{\#\textgreater{} [15] "libloc\_195\_e5e4a02699cec889.rds"                            }
\CommentTok{\#\textgreater{} [16] "report.mztab"                                               }
\CommentTok{\#\textgreater{} [17] "repos\_https\%3A\%2F\%2Fcloud.r{-}project.org\%2Fsrc\%2Fcontrib.rds"}
\end{Highlighting}
\end{Shaded}

\hypertarget{data-collating}{%
\paragraph{Data collating}\label{data-collating}}

To begin with, MCnebula should he initialized in SIRIUS project.

\begin{Shaded}
\begin{Highlighting}[]
\NormalTok{MCnebula}\OperatorTok{::}\KeywordTok{initialize\_mcnebula}\NormalTok{(tmp, }\DataTypeTok{rm\_mc.set =}\NormalTok{ T)}
\CommentTok{\#\textgreater{} MCnebula project has initialized at {-}\textgreater{} /tmp/Rtmp12SxNi}
\end{Highlighting}
\end{Shaded}

This will set some global var.

\begin{Shaded}
\begin{Highlighting}[]
\KeywordTok{ls}\NormalTok{(}\DataTypeTok{pattern =} \StringTok{"\^{}}\CharTok{\textbackslash{}\textbackslash{}}\StringTok{.MCn}\CharTok{\textbackslash{}\textbackslash{}}\StringTok{."}\NormalTok{, }\DataTypeTok{all.names =}\NormalTok{ T)}
\CommentTok{\#\textgreater{} [1] ".MCn.output"        ".MCn.palette"       ".MCn.palette\_label"}
\CommentTok{\#\textgreater{} [4] ".MCn.palette\_ppcp"  ".MCn.palette\_stat"  ".MCn.results"      }
\CommentTok{\#\textgreater{} [7] ".MCn.sirius"}
\end{Highlighting}
\end{Shaded}

Users can manually modify the setup.

\hypertarget{collate-structure}{%
\subparagraph{Collate structure}\label{collate-structure}}

\begin{Shaded}
\begin{Highlighting}[]
\NormalTok{MCnebula}\OperatorTok{::}\KeywordTok{collate\_structure}\NormalTok{(}
  \DataTypeTok{exclude\_element =} \KeywordTok{c}\NormalTok{(}\StringTok{"Cl"}\NormalTok{, }\StringTok{"S"}\NormalTok{, }\StringTok{"P"}\NormalTok{),}
  \DataTypeTok{ppm\_error =} \DecValTok{20}
\NormalTok{)}
\CommentTok{\#\textgreater{} [INFO] MCnebula run: collate\_structure}
\CommentTok{\#\textgreater{} \#\# collate\_structure: check\_dir}
\CommentTok{\#\textgreater{} \#\# collate\_structure: method\_pick\_formula\_excellent}
\CommentTok{\#\textgreater{} \#\# Method part: batch\_get\_formula}
\CommentTok{\#\textgreater{} \#\# Method part: batch\_get\_structure}
\CommentTok{\#\textgreater{} \#\# collate\_structure: re{-}collate structure}
\CommentTok{\#\textgreater{} [INFO] MCnebula Job Done: collate\_structure.}
\end{Highlighting}
\end{Shaded}

The results:

\begin{Shaded}
\begin{Highlighting}[]
\NormalTok{.MCn.formula\_set}
\CommentTok{\#\textgreater{} \# A tibble: 5 x 8}
\CommentTok{\#\textgreater{}   .id       rank precursorFormula molecularFormula adduct   ZodiacScore}
\CommentTok{\#\textgreater{}   \textless{}chr\textgreater{}    \textless{}dbl\textgreater{} \textless{}chr\textgreater{}            \textless{}chr\textgreater{}            \textless{}chr\textgreater{}          \textless{}dbl\textgreater{}}
\CommentTok{\#\textgreater{} 1 gnps1234     1 C25H41NO7        C25H41NO7        [M + H]+       0.551}
\CommentTok{\#\textgreater{} 2 gnps1537     1 C33H47N5O6       C33H47N5O6       [M + H]+       0    }
\CommentTok{\#\textgreater{} 3 gnps1539     1 C33H47N5O6       C33H47N5O6       [M + H]+       0    }
\CommentTok{\#\textgreater{} 4 gnps1567     1 C32H32N4O4       C32H32N4O4       [M + H]+       0    }
\CommentTok{\#\textgreater{} 5 gnps1588     1 C24H17N3O2       C24H17N3O2       [M + H]+       0    }
\CommentTok{\#\textgreater{} \# ... with 2 more variables: \textasciigrave{}massErrorPrecursor(ppm)\textasciigrave{} \textless{}dbl\textgreater{}, use\_zodiac \textless{}lgl\textgreater{}}
\NormalTok{.MCn.structure\_set}
\CommentTok{\#\textgreater{} \# A tibble: 5 x 15}
\CommentTok{\#\textgreater{}   .id      file\_name  inchikey2D inchi molecularFormula  rank score name  smiles}
\CommentTok{\#\textgreater{}   \textless{}chr\textgreater{}    \textless{}chr\textgreater{}      \textless{}chr\textgreater{}      \textless{}chr\textgreater{} \textless{}chr\textgreater{}            \textless{}int\textgreater{} \textless{}dbl\textgreater{} \textless{}chr\textgreater{} \textless{}chr\textgreater{} }
\CommentTok{\#\textgreater{} 1 gnps1234 C25H41NO7\textasciitilde{} YPJNTJHXP\textasciitilde{} InCh\textasciitilde{} C25H41NO7            1 {-}37.3 null  CCN1C\textasciitilde{}}
\CommentTok{\#\textgreater{} 2 gnps1537 C33H47N5O\textasciitilde{} HZWIJOWMG\textasciitilde{} InCh\textasciitilde{} C33H47N5O6           1 {-}22.2 (6S,\textasciitilde{} CCC(C\textasciitilde{}}
\CommentTok{\#\textgreater{} 3 gnps1539 C33H47N5O\textasciitilde{} GWLHFNKLJ\textasciitilde{} InCh\textasciitilde{} C33H47N5O6           1 {-}19.9 null  CCC(=\textasciitilde{}}
\CommentTok{\#\textgreater{} 4 gnps1567 C32H32N4O\textasciitilde{} BUTLVTAFD\textasciitilde{} InCh\textasciitilde{} C32H32N4O4           1 {-}21.2 null  CC1(C\textasciitilde{}}
\CommentTok{\#\textgreater{} 5 gnps1588 C24H17N3O\textasciitilde{} HXDZMNFJQ\textasciitilde{} InCh\textasciitilde{} C24H17N3O2           1 {-}18.1 Benz\textasciitilde{} CN1C(\textasciitilde{}}
\CommentTok{\#\textgreater{} \# ... with 6 more variables: xlogp \textless{}dbl\textgreater{}, PubMedIds \textless{}chr\textgreater{}, links \textless{}chr\textgreater{},}
\CommentTok{\#\textgreater{} \#   tanimotoSimilarity \textless{}dbl\textgreater{}, dbflags \textless{}int64\textgreater{}, structure\_rank \textless{}int\textgreater{}}
\end{Highlighting}
\end{Shaded}

Next, we collate the metadata of chemical classes hierarchy.

\begin{Shaded}
\begin{Highlighting}[]
\NormalTok{MCnebula}\OperatorTok{::}\KeywordTok{build\_classes\_tree\_list}\NormalTok{()}
\CommentTok{\#\textgreater{} INFO: Classification Index in.MCn.sirius project {-}{-}{-}\textgreater{} canopus.tsv }
\CommentTok{\#\textgreater{} A total of 11 levels.}
\CommentTok{\#\textgreater{} These classes (upon ClassyFire and CANOPUS) were separated into sub{-}lists.}
\CommentTok{\#\textgreater{}         Use following arguments to get some specific classes:}
\CommentTok{\#\textgreater{}         .MCn.class\_tree\_list[[3]] \textgreater{}\textgreater{}\textgreater{} superclass}
\CommentTok{\#\textgreater{}         .MCn.class\_tree\_list[[4]] \textgreater{}\textgreater{}\textgreater{} class}
\CommentTok{\#\textgreater{}         .MCn.class\_tree\_list[[5]] \textgreater{}\textgreater{}\textgreater{} subclass}
\CommentTok{\#\textgreater{}         .MCn.class\_tree\_list[[6]] \textgreater{}\textgreater{}\textgreater{} level 5}
\end{Highlighting}
\end{Shaded}

The frame of returned project, e.g.:

\begin{Shaded}
\begin{Highlighting}[]
\NormalTok{.MCn.class\_tree\_list[[}\DecValTok{4}\NormalTok{]]}
\CommentTok{\#\textgreater{} \# A tibble: 442 x 6}
\CommentTok{\#\textgreater{}    relativeIndex absoluteIndex id              name         parentId description}
\CommentTok{\#\textgreater{}            \textless{}int\textgreater{}         \textless{}int\textgreater{} \textless{}chr\textgreater{}           \textless{}chr\textgreater{}        \textless{}chr\textgreater{}    \textless{}chr\textgreater{}      }
\CommentTok{\#\textgreater{}  1             3             3 CHEMONT:0000003 Thiepanes    CHEMONT\textasciitilde{} Organic co\textasciitilde{}}
\CommentTok{\#\textgreater{}  2             5             6 CHEMONT:0000006 Benzoxazoli\textasciitilde{} CHEMONT\textasciitilde{} Organic co\textasciitilde{}}
\CommentTok{\#\textgreater{}  3            15            18 CHEMONT:0000018 Anthracenes  CHEMONT\textasciitilde{} Organic co\textasciitilde{}}
\CommentTok{\#\textgreater{}  4            16            19 CHEMONT:0000019 Dibenzocycl\textasciitilde{} CHEMONT\textasciitilde{} Compounds \textasciitilde{}}
\CommentTok{\#\textgreater{}  5            17            20 CHEMONT:0000020 Fluorenes    CHEMONT\textasciitilde{} Compounds \textasciitilde{}}
\CommentTok{\#\textgreater{}  6            18            21 CHEMONT:0000021 Indenes and\textasciitilde{} CHEMONT\textasciitilde{} Compounds \textasciitilde{}}
\CommentTok{\#\textgreater{}  7            19            22 CHEMONT:0000022 Naphthacenes CHEMONT\textasciitilde{} Compounds \textasciitilde{}}
\CommentTok{\#\textgreater{}  8            20            23 CHEMONT:0000023 Naphthalenes CHEMONT\textasciitilde{} Compounds \textasciitilde{}}
\CommentTok{\#\textgreater{}  9            21            24 CHEMONT:0000024 Pentacenes   CHEMONT\textasciitilde{} Compounds \textasciitilde{}}
\CommentTok{\#\textgreater{} 10            22            25 CHEMONT:0000025 Phenanthren\textasciitilde{} CHEMONT\textasciitilde{} Polycyclic\textasciitilde{}}
\CommentTok{\#\textgreater{} \# ... with 432 more rows}
\end{Highlighting}
\end{Shaded}

\hypertarget{collate-ppcp-data.}{%
\subparagraph{Collate PPCP data.}\label{collate-ppcp-data.}}

This function retrieve all posterior probability of classification
prediction (PPCP) of compounds and gathered them as
\texttt{.MCn.ppcp\_dataset}. Furthermore, in default, this function also
summarise \texttt{.MCn.nebula\_class} and \texttt{.MCn.nebula\_index}
upon the data of \texttt{.MCn.ppcp\_dataset}. If done,
\texttt{.MCn.nebula\_index} would be returned.

\begin{Shaded}
\begin{Highlighting}[]
\NormalTok{MCnebula}\OperatorTok{::}\KeywordTok{collate\_ppcp}\NormalTok{(}
  \CommentTok{\#\# due to the size of example dataset, we set much smaller herein}
  \CommentTok{\#\# min\_possess, 20 or more may be better}
  \DataTypeTok{min\_possess =} \DecValTok{1}\NormalTok{,}
  \CommentTok{\#\# 0.1 may better}
  \DataTypeTok{max\_possess\_pct =} \FloatTok{0.9}
\NormalTok{)}
\CommentTok{\#\textgreater{} [INFO] MCnebula run: collate\_ppcp}
\CommentTok{\#\textgreater{} \#\# collate\_ppcp: check\_dir}
\CommentTok{\#\textgreater{} \#\# STAT of PPCP dataset: 5(formula with PPCP)/5(all formula) }
\CommentTok{\#\textgreater{} \#\# collate\_ppcp: method\_summarize\_nebula\_class}
\CommentTok{\#\textgreater{} \#\# collate\_ppcp: method\_summarize\_nebula\_index.}
\CommentTok{\#\textgreater{} \#\# Method part: class\_retrieve}
\CommentTok{\#\textgreater{} \#\# Method part: identical\_filter}
\CommentTok{\#\textgreater{} \#\# Method part: fun\_filter\_via\_struc\_score}
\CommentTok{\#\textgreater{} [INFO] MCnebula Job Done: collate\_ppcp.}
\CommentTok{\#\textgreater{} \# A tibble: 25 x 5}
\CommentTok{\#\textgreater{} \# Groups:   relativeIndex [12]}
\CommentTok{\#\textgreater{}    relativeIndex name                             hierarchy .id         V1}
\CommentTok{\#\textgreater{}            \textless{}int\textgreater{} \textless{}chr\textgreater{}                                \textless{}int\textgreater{} \textless{}chr\textgreater{}    \textless{}dbl\textgreater{}}
\CommentTok{\#\textgreater{}  1             9 Lipids and lipid{-}like molecules          3 gnps1234 0.995}
\CommentTok{\#\textgreater{}  2           233 Phenylpropanoids and polyketides         3 gnps1537 0.998}
\CommentTok{\#\textgreater{}  3           233 Phenylpropanoids and polyketides         3 gnps1539 0.999}
\CommentTok{\#\textgreater{}  4           235 Organic acids and derivatives            3 gnps1537 0.999}
\CommentTok{\#\textgreater{}  5           235 Organic acids and derivatives            3 gnps1539 1.00 }
\CommentTok{\#\textgreater{}  6           235 Organic acids and derivatives            3 gnps1567 0.997}
\CommentTok{\#\textgreater{}  7           235 Organic acids and derivatives            3 gnps1588 0.990}
\CommentTok{\#\textgreater{}  8           247 Alkaloids and derivatives                3 gnps1234 0.998}
\CommentTok{\#\textgreater{}  9          1298 Benzenoids                               3 gnps1537 0.999}
\CommentTok{\#\textgreater{} 10          1298 Benzenoids                               3 gnps1539 0.999}
\CommentTok{\#\textgreater{} \# ... with 15 more rows}
\end{Highlighting}
\end{Shaded}

In addition, \texttt{collate\_ppcp} would set following global var:

\begin{Shaded}
\begin{Highlighting}[]
\KeywordTok{head}\NormalTok{(.MCn.ppcp\_dataset, }\DataTypeTok{n =} \DecValTok{1}\NormalTok{)}
\CommentTok{\#\textgreater{} $gnps1234}
\CommentTok{\#\textgreater{}                 V1 relativeIndex}
\CommentTok{\#\textgreater{}    1: 9.999967e{-}01             0}
\CommentTok{\#\textgreater{}    2: 1.175820e{-}05             1}
\CommentTok{\#\textgreater{}    3: 9.987737e{-}01             2}
\CommentTok{\#\textgreater{}    4: 8.464927e{-}10             3}
\CommentTok{\#\textgreater{}    5: 3.867385e{-}10             4}
\CommentTok{\#\textgreater{}   {-}{-}{-}                           }
\CommentTok{\#\textgreater{} 2505: 1.825751e{-}11          2504}
\CommentTok{\#\textgreater{} 2506: 1.821264e{-}13          2505}
\CommentTok{\#\textgreater{} 2507: 2.369786e{-}15          2506}
\CommentTok{\#\textgreater{} 2508: 7.240430e{-}12          2507}
\CommentTok{\#\textgreater{} 2509: 9.999823e{-}01          2508}
\KeywordTok{head}\NormalTok{(.MCn.nebula\_class, }\DataTypeTok{n =} \DecValTok{1}\NormalTok{) }\OperatorTok{\%\textgreater{}\%}\StringTok{ }
\StringTok{  }\KeywordTok{lapply}\NormalTok{(dplyr}\OperatorTok{::}\NormalTok{as\_tibble)}
\CommentTok{\#\textgreater{} $gnps1234}
\CommentTok{\#\textgreater{} \# A tibble: 26 x 6}
\CommentTok{\#\textgreater{}    relativeIndex    V1 hierarchy absoluteIndex id              name             }
\CommentTok{\#\textgreater{}            \textless{}int\textgreater{} \textless{}dbl\textgreater{}     \textless{}int\textgreater{}         \textless{}int\textgreater{} \textless{}chr\textgreater{}           \textless{}chr\textgreater{}            }
\CommentTok{\#\textgreater{}  1          2012 1.00          6          3781 CHEMONT:0003788 Aconitane{-}type d\textasciitilde{}}
\CommentTok{\#\textgreater{}  2          1302 1.00          6          2451 CHEMONT:0002452 Tertiary amines  }
\CommentTok{\#\textgreater{}  3           593 0.999         6          1292 CHEMONT:0001292 Cyclic alcohols \textasciitilde{}}
\CommentTok{\#\textgreater{}  4           512 0.995         6          1167 CHEMONT:0001167 Dialkyl ethers   }
\CommentTok{\#\textgreater{}  5           834 0.971         6          1669 CHEMONT:0001670 Tertiary alcohols}
\CommentTok{\#\textgreater{}  6          1191 0.962         6          2285 CHEMONT:0002286 Polyols          }
\CommentTok{\#\textgreater{}  7           826 0.958         6          1660 CHEMONT:0001661 Secondary alcoho\textasciitilde{}}
\CommentTok{\#\textgreater{}  8          1309 0.941         6          2459 CHEMONT:0002460 Alkanolamines    }
\CommentTok{\#\textgreater{}  9          1299 1.00          5          2448 CHEMONT:0002449 Amines           }
\CommentTok{\#\textgreater{} 10           116 1.00          5           129 CHEMONT:0000129 Alcohols and pol\textasciitilde{}}
\CommentTok{\#\textgreater{} \# ... with 16 more rows}
\end{Highlighting}
\end{Shaded}

   Indeed, the above prepared instance data is too small to show the
detail of the following function of MCnebula. Hence, we used another
projects which has been collated done via above functions.

\begin{Shaded}
\begin{Highlighting}[]
\CommentTok{\#\#\# the following project were auto{-}loaded while use \textquotesingle{}library(MCnebula)\textquotesingle{}}
\NormalTok{inst\_formula\_set}
\CommentTok{\#\textgreater{} \# A tibble: 700 x 8}
\CommentTok{\#\textgreater{}    .id       rank precursorFormula molecularFormula adduct    ZodiacScore}
\CommentTok{\#\textgreater{}    \textless{}chr\textgreater{}    \textless{}dbl\textgreater{} \textless{}chr\textgreater{}            \textless{}chr\textgreater{}            \textless{}chr\textgreater{}           \textless{}dbl\textgreater{}}
\CommentTok{\#\textgreater{}  1 gnps1588     1 C24H17N3O2       C24H17N3O2       [M + H]+        0    }
\CommentTok{\#\textgreater{}  2 gnps201      1 C17H20N2O5S      C17H20N2O5S      [M + H]+        0    }
\CommentTok{\#\textgreater{}  3 gnps2137     1 C35H46O11        C35H46O11        [M + K]+        0.699}
\CommentTok{\#\textgreater{}  4 gnps2588     1 C22H26O11        C22H26O11        [M + H]+        0    }
\CommentTok{\#\textgreater{}  5 gnps2679     1 C32H38N2O6       C32H38N2O6       [M + H]+        0    }
\CommentTok{\#\textgreater{}  6 gnps2731     1 C30H28O12        C30H28O12        [M + H]+        0    }
\CommentTok{\#\textgreater{}  7 gnps2826     1 C32H56O13        C32H56O13        [M + H]+        0    }
\CommentTok{\#\textgreater{}  8 gnps3071     1 C17H20O5         C17H20O5         [M + Na]+       0    }
\CommentTok{\#\textgreater{}  9 gnps3116     1 C20H21NO4        C20H21NO4        [M + H]+        0    }
\CommentTok{\#\textgreater{} 10 gnps3638     1 C19H22O5         C19H22O5         [M + H]+        0    }
\CommentTok{\#\textgreater{} \# ... with 690 more rows, and 2 more variables:}
\CommentTok{\#\textgreater{} \#   \textasciigrave{}massErrorPrecursor(ppm)\textasciigrave{} \textless{}dbl\textgreater{}, use\_zodiac \textless{}lgl\textgreater{}}
\NormalTok{inst\_structure\_set}
\CommentTok{\#\textgreater{} \# A tibble: 700 x 7}
\CommentTok{\#\textgreater{}    .id      file\_name         score smiles tanimotoSimilar\textasciitilde{} structure\_rank name }
\CommentTok{\#\textgreater{}    \textless{}chr\textgreater{}    \textless{}chr\textgreater{}             \textless{}dbl\textgreater{} \textless{}chr\textgreater{}             \textless{}dbl\textgreater{}          \textless{}dbl\textgreater{} \textless{}chr\textgreater{}}
\CommentTok{\#\textgreater{}  1 gnps4260 C18H26N2O9\_[M+H]+ {-}34.3 CC(C(\textasciitilde{}            0.944              1 null }
\CommentTok{\#\textgreater{}  2 gnps7278 C23H36O6\_[M+H]+   {-}32.2 CCC(C\textasciitilde{}            0.921              1 null }
\CommentTok{\#\textgreater{}  3 gnps3385 C15H15NO3\_[M+H]+  {-}75.4 C1=CC\textasciitilde{}            0.613              1 null }
\CommentTok{\#\textgreater{}  4 gnps1779 C16H17NO3\_[M+H]+  {-}14.5 C1CCN\textasciitilde{}            0.989              1 null }
\CommentTok{\#\textgreater{}  5 gnps7175 C8H15N5S\_[M+H]+   {-}11.4 CCNC1\textasciitilde{}            1                  1 null }
\CommentTok{\#\textgreater{}  6 gnps236  C18H20N2\_[M+H]+   {-}10.5 CN1CC\textasciitilde{}            1                  1 null }
\CommentTok{\#\textgreater{}  7 gnps2588 C22H26O11\_[M+H]+  {-}13.6 CC(C)\textasciitilde{}            1                  1 null }
\CommentTok{\#\textgreater{}  8 gnps5852 C33H52O13\_[M+Na]+ {-}52.0 CC12C\textasciitilde{}            0.843              1 null }
\CommentTok{\#\textgreater{}  9 gnps4461 C12H14O5\_[M+H]+   {-}37.0 CCOC(\textasciitilde{}            0.854              1 null }
\CommentTok{\#\textgreater{} 10 gnps3779 C23H22O13\_[M+H]+  {-}11.7 CC(=O\textasciitilde{}            1                  1 null }
\CommentTok{\#\textgreater{} \# ... with 690 more rows}
\KeywordTok{head}\NormalTok{(inst\_ppcp\_dataset, }\DataTypeTok{n =} \DecValTok{1}\NormalTok{)}
\CommentTok{\#\textgreater{} $gnps106}
\CommentTok{\#\textgreater{}         V1 relativeIndex}
\CommentTok{\#\textgreater{}    1: 1.00             0}
\CommentTok{\#\textgreater{}    2: 0.00             1}
\CommentTok{\#\textgreater{}    3: 0.01             2}
\CommentTok{\#\textgreater{}    4: 0.00             3}
\CommentTok{\#\textgreater{}    5: 0.00             4}
\CommentTok{\#\textgreater{}   {-}{-}{-}                   }
\CommentTok{\#\textgreater{} 2505: 0.00          2504}
\CommentTok{\#\textgreater{} 2506: 0.00          2505}
\CommentTok{\#\textgreater{} 2507: 0.00          2506}
\CommentTok{\#\textgreater{} 2508: 0.00          2507}
\CommentTok{\#\textgreater{} 2509: 1.00          2508}
\end{Highlighting}
\end{Shaded}

We set these data as global var, so as the MCnebula function could
recognized them without additional parameters.

\begin{Shaded}
\begin{Highlighting}[]
\NormalTok{.MCn.formula\_set \textless{}{-}}\StringTok{ }\NormalTok{inst\_formula\_set}
\NormalTok{.MCn.structure\_set \textless{}{-}}\StringTok{ }\NormalTok{inst\_structure\_set}
\NormalTok{.MCn.ppcp\_dataset \textless{}{-}}\StringTok{ }\NormalTok{inst\_ppcp\_dataset}
\end{Highlighting}
\end{Shaded}

Re-execute the command:

\begin{Shaded}
\begin{Highlighting}[]
\NormalTok{MCnebula}\OperatorTok{::}\KeywordTok{collate\_ppcp}\NormalTok{(}
  \DataTypeTok{min\_possess =} \DecValTok{20}\NormalTok{,}
  \DataTypeTok{max\_possess\_pct =} \FloatTok{0.1}
\NormalTok{)}
\CommentTok{\#\textgreater{} [INFO] MCnebula run: collate\_ppcp}
\CommentTok{\#\textgreater{} \#\# collate\_ppcp: check\_dir}
\CommentTok{\#\textgreater{} \#\# STAT of PPCP dataset: 1(formula with PPCP)/5(all formula) }
\CommentTok{\#\textgreater{} \#\# collate\_ppcp: method\_summarize\_nebula\_class}
\CommentTok{\#\textgreater{} \#\# collate\_ppcp: method\_summarize\_nebula\_index.}
\CommentTok{\#\textgreater{} \#\# Method part: class\_retrieve}
\CommentTok{\#\textgreater{} \#\# Method part: identical\_filter}
\CommentTok{\#\textgreater{} \#\# Method part: fun\_filter\_via\_struc\_score}
\CommentTok{\#\textgreater{} [INFO] MCnebula Job Done: collate\_ppcp.}
\CommentTok{\#\textgreater{} \# A tibble: 1,332 x 5}
\CommentTok{\#\textgreater{} \# Groups:   relativeIndex [38]}
\CommentTok{\#\textgreater{}    relativeIndex name                   hierarchy .id         V1}
\CommentTok{\#\textgreater{}            \textless{}int\textgreater{} \textless{}chr\textgreater{}                      \textless{}int\textgreater{} \textless{}chr\textgreater{}    \textless{}dbl\textgreater{}}
\CommentTok{\#\textgreater{}  1             4 Organosulfur compounds         3 gnps1816  1   }
\CommentTok{\#\textgreater{}  2             4 Organosulfur compounds         3 gnps1914  1   }
\CommentTok{\#\textgreater{}  3             4 Organosulfur compounds         3 gnps201   1   }
\CommentTok{\#\textgreater{}  4             4 Organosulfur compounds         3 gnps449   0.99}
\CommentTok{\#\textgreater{}  5             4 Organosulfur compounds         3 gnps469   0.97}
\CommentTok{\#\textgreater{}  6             4 Organosulfur compounds         3 gnps494   0.98}
\CommentTok{\#\textgreater{}  7             4 Organosulfur compounds         3 gnps4969  1   }
\CommentTok{\#\textgreater{}  8             4 Organosulfur compounds         3 gnps5242  0.99}
\CommentTok{\#\textgreater{}  9             4 Organosulfur compounds         3 gnps5693  0.99}
\CommentTok{\#\textgreater{} 10             4 Organosulfur compounds         3 gnps5917  0.99}
\CommentTok{\#\textgreater{} \# ... with 1,322 more rows}
\end{Highlighting}
\end{Shaded}

\hypertarget{generate-network-graph}{%
\paragraph{Generate network graph}\label{generate-network-graph}}

\hypertarget{generate-parent-nebula-graph}{%
\subparagraph{Generate parent-nebula
graph}\label{generate-parent-nebula-graph}}

Of note, this function will conduct spectral similarity (fragmentation
spectra) computation and reformated these similarity (cosine) as edges
file. This is performed via \texttt{MSnbase::compareSpectra}. It usually
cost some time to get results. Here, we used a randomly formed edges
file to avoid it.

\begin{Shaded}
\begin{Highlighting}[]
\NormalTok{edges \textless{}{-}}\StringTok{ }\NormalTok{.MCn.formula\_set}\OperatorTok{$}\NormalTok{.id }\OperatorTok{\%\textgreater{}\%}\StringTok{ }
\StringTok{  }\KeywordTok{combn}\NormalTok{(}\DecValTok{2}\NormalTok{) }\OperatorTok{\%\textgreater{}\%}\StringTok{ }
\StringTok{  }\KeywordTok{t}\NormalTok{()}
\KeywordTok{set.seed}\NormalTok{(}\DecValTok{100}\NormalTok{)}
\CommentTok{\#\#\# randomly get combination.}
\NormalTok{edges \textless{}{-}}\StringTok{ }\NormalTok{edges[}\KeywordTok{sample}\NormalTok{(}\DecValTok{1}\OperatorTok{:}\KeywordTok{nrow}\NormalTok{(edges), }\DecValTok{1000}\NormalTok{), ] }\OperatorTok{\%\textgreater{}\%}\StringTok{ }
\StringTok{  }\NormalTok{dplyr}\OperatorTok{::}\KeywordTok{as\_tibble}\NormalTok{() }\OperatorTok{\%\textgreater{}\%}\StringTok{ }
\StringTok{  }\NormalTok{dplyr}\OperatorTok{::}\KeywordTok{rename}\NormalTok{(}
    \DataTypeTok{.id\_1 =} \DecValTok{1}\NormalTok{,}
    \DataTypeTok{.id\_2 =} \DecValTok{2}\NormalTok{) }\OperatorTok{\%\textgreater{}\%}\StringTok{ }
\StringTok{  }\NormalTok{dplyr}\OperatorTok{::}\KeywordTok{mutate}\NormalTok{(}
    \DataTypeTok{dotproduct =} \KeywordTok{rnorm}\NormalTok{(}\DecValTok{1000}\NormalTok{, }\DataTypeTok{mean =} \FloatTok{0.5}\NormalTok{, }\DataTypeTok{sd =} \FloatTok{0.1}\NormalTok{), }
    \DataTypeTok{mass\_diff =} \KeywordTok{rnorm}\NormalTok{(}\DecValTok{1000}\NormalTok{, }\DataTypeTok{mean =} \DecValTok{200}\NormalTok{, }\DataTypeTok{sd =} \DecValTok{50}\NormalTok{),}
\NormalTok{  )}
\NormalTok{tmp.edges \textless{}{-}}\StringTok{ }\KeywordTok{tempfile}\NormalTok{()}
\CommentTok{\#\#\# write done}
\NormalTok{MCnebula}\OperatorTok{::}\KeywordTok{write\_tsv}\NormalTok{(edges, }\DataTypeTok{filename =}\NormalTok{ tmp.edges)}
\CommentTok{\#\#\# show details}
\NormalTok{edges}
\CommentTok{\#\textgreater{} \# A tibble: 1,000 x 4}
\CommentTok{\#\textgreater{}    .id\_1    .id\_2    dotproduct mass\_diff}
\CommentTok{\#\textgreater{}    \textless{}chr\textgreater{}    \textless{}chr\textgreater{}         \textless{}dbl\textgreater{}     \textless{}dbl\textgreater{}}
\CommentTok{\#\textgreater{}  1 gnps2299 gnps8202      0.791     150. }
\CommentTok{\#\textgreater{}  2 gnps2056 gnps3366      0.503     224. }
\CommentTok{\#\textgreater{}  3 gnps4151 gnps8192      0.414     199. }
\CommentTok{\#\textgreater{}  4 gnps2468 gnps2494      0.575     129. }
\CommentTok{\#\textgreater{}  5 gnps5852 gnps8615      0.604      90.1}
\CommentTok{\#\textgreater{}  6 gnps2503 gnps4589      0.742     212. }
\CommentTok{\#\textgreater{}  7 gnps4564 gnps4790      0.650     128. }
\CommentTok{\#\textgreater{}  8 gnps2437 gnps5779      0.424     256. }
\CommentTok{\#\textgreater{}  9 gnps4589 gnps8087      0.530     223. }
\CommentTok{\#\textgreater{} 10 gnps4661 gnps6791      0.543     228. }
\CommentTok{\#\textgreater{} \# ... with 990 more rows}
\end{Highlighting}
\end{Shaded}

Subsequently, we use this edges file to generate parent-nebula graph.

\begin{Shaded}
\begin{Highlighting}[]
\NormalTok{MCnebula}\OperatorTok{::}\KeywordTok{generate\_parent\_nebula}\NormalTok{(}
  \DataTypeTok{rm\_parent\_isolate\_nodes =}\NormalTok{ T,}
  \CommentTok{\#\# specify the edges file}
  \DataTypeTok{edges\_file =}\NormalTok{ tmp.edges}
\NormalTok{)}
\CommentTok{\#\textgreater{} [INFO] MCnebula run: generate\_parent\_nebula}
\CommentTok{\#\textgreater{} \#\# generate\_parent\_nebula: file.exists(edges\_file) == T.}
\CommentTok{\#\textgreater{} Escape from time{-}consuming computation}
\CommentTok{\#\textgreater{} [INFO] MCnebula Job Done: generate\_parent\_nebula}
\end{Highlighting}
\end{Shaded}

This function will generate nodes data (chemical formula and strucutre
annotation) and edges data (if the edges file were not specified but
real-time computed). For most, parent-nebula graph was generated.

\begin{Shaded}
\begin{Highlighting}[]
\NormalTok{.MCn.parent\_nodes}
\CommentTok{\#\textgreater{} \# A tibble: 700 x 14}
\CommentTok{\#\textgreater{}    .id       rank precursorFormula molecularFormula adduct    ZodiacScore}
\CommentTok{\#\textgreater{}    \textless{}chr\textgreater{}    \textless{}dbl\textgreater{} \textless{}chr\textgreater{}            \textless{}chr\textgreater{}            \textless{}chr\textgreater{}           \textless{}dbl\textgreater{}}
\CommentTok{\#\textgreater{}  1 gnps106      1 C11H12O5         C11H12O5         [M + H]+        1    }
\CommentTok{\#\textgreater{}  2 gnps108      1 C17H35NO2        C17H35NO2        [M + H]+        1    }
\CommentTok{\#\textgreater{}  3 gnps1146     1 C19H22O11        C19H22O11        [M + Na]+       0.880}
\CommentTok{\#\textgreater{}  4 gnps1153     1 C12H21N          C12H21N          [M + H]+        1    }
\CommentTok{\#\textgreater{}  5 gnps1167     1 C22H32O3         C22H32O3         [M + H]+        1    }
\CommentTok{\#\textgreater{}  6 gnps1168     1 C21H26O5         C21H26O5         [M + H]+        1    }
\CommentTok{\#\textgreater{}  7 gnps1170     1 C24H34O4         C24H34O4         [M + H]+        1    }
\CommentTok{\#\textgreater{}  8 gnps1171     1 C21H30O2         C21H30O2         [M + H]+        1    }
\CommentTok{\#\textgreater{}  9 gnps1175     1 C22H38O5         C22H38O5         [M + Na]+       1    }
\CommentTok{\#\textgreater{} 10 gnps1176     1 C21H28O5         C21H28O5         [M + H]+        1    }
\CommentTok{\#\textgreater{} \# ... with 690 more rows, and 8 more variables:}
\CommentTok{\#\textgreater{} \#   \textasciigrave{}massErrorPrecursor(ppm)\textasciigrave{} \textless{}dbl\textgreater{}, use\_zodiac \textless{}lgl\textgreater{}, file\_name \textless{}chr\textgreater{},}
\CommentTok{\#\textgreater{} \#   score \textless{}dbl\textgreater{}, smiles \textless{}chr\textgreater{}, tanimotoSimilarity \textless{}dbl\textgreater{}, structure\_rank \textless{}dbl\textgreater{},}
\CommentTok{\#\textgreater{} \#   compound\_name \textless{}chr\textgreater{}}
\NormalTok{.MCn.parent\_graph}
\CommentTok{\#\textgreater{} IGRAPH 8f379f2 DN{-}{-} 661 1000 {-}{-} }
\CommentTok{\#\textgreater{} + attr: name (v/c), rank (v/n), precursorFormula (v/c),}
\CommentTok{\#\textgreater{} | molecularFormula (v/c), adduct (v/c), ZodiacScore (v/n),}
\CommentTok{\#\textgreater{} | massErrorPrecursor(ppm) (v/n), use\_zodiac (v/l), file\_name (v/c),}
\CommentTok{\#\textgreater{} | score (v/n), smiles (v/c), tanimotoSimilarity (v/n), structure\_rank}
\CommentTok{\#\textgreater{} | (v/n), compound\_name (v/c), dotproduct (e/n), mass\_diff (e/n)}
\CommentTok{\#\textgreater{} + edges from 8f379f2 (vertex names):}
\CommentTok{\#\textgreater{}  [1] gnps2299{-}\textgreater{}gnps8202 gnps2056{-}\textgreater{}gnps3366 gnps4151{-}\textgreater{}gnps8192 gnps2468{-}\textgreater{}gnps2494}
\CommentTok{\#\textgreater{}  [5] gnps5852{-}\textgreater{}gnps8615 gnps2503{-}\textgreater{}gnps4589 gnps4564{-}\textgreater{}gnps4790 gnps2437{-}\textgreater{}gnps5779}
\CommentTok{\#\textgreater{}  [9] gnps4589{-}\textgreater{}gnps8087 gnps4661{-}\textgreater{}gnps6791 gnps2966{-}\textgreater{}gnps633  gnps5860{-}\textgreater{}gnps6479}
\CommentTok{\#\textgreater{} [13] gnps4898{-}\textgreater{}gnps7107 gnps2345{-}\textgreater{}gnps3674 gnps3268{-}\textgreater{}gnps4120 gnps3164{-}\textgreater{}gnps8482}
\CommentTok{\#\textgreater{} + ... omitted several edges}
\end{Highlighting}
\end{Shaded}

\hypertarget{generate-child-nebulae-graph}{%
\subparagraph{Generate child-nebulae
graph}\label{generate-child-nebulae-graph}}

This function generate multiple network graph upon parent-nebula (nodes
and edges) according to compound classification.

\begin{Shaded}
\begin{Highlighting}[]
\NormalTok{MCnebula}\OperatorTok{::}\KeywordTok{generate\_child\_nebulae}\NormalTok{(}
  \DataTypeTok{max\_edges =} \DecValTok{5}\NormalTok{,}
  \CommentTok{\#\# this will write a output of .graphml, which support by Cytoscape.}
  \DataTypeTok{output\_format =} \StringTok{"graphml"}
\NormalTok{)}
\CommentTok{\#\textgreater{} [INFO] MCnebula run: generate\_child\_nebulae}
\CommentTok{\#\textgreater{} [INFO] MCnebula Job Done: generate\_child\_nebulae}
\end{Highlighting}
\end{Shaded}

These graph were stored in list.

\begin{verbatim}
head(.MCn.child_graph_list, n = 2)
\end{verbatim}

\hypertarget{visualization-of-chemical-nebula}{%
\paragraph{Visualization of
chemical-nebula}\label{visualization-of-chemical-nebula}}

\hypertarget{visualization-of-parent-nebula}{%
\subparagraph{Visualization of
parent-nebula}\label{visualization-of-parent-nebula}}

All visualization in MCnebula package are output with .svg image.

\begin{Shaded}
\begin{Highlighting}[]
\NormalTok{MCnebula}\OperatorTok{::}\KeywordTok{visualize\_parent\_nebula}\NormalTok{(}
  \DataTypeTok{layout =} \StringTok{"kk"}\NormalTok{,}
  \CommentTok{\#\# map nodes color with superclass}
  \DataTypeTok{nodes\_color =} \KeywordTok{c}\NormalTok{(}\StringTok{"hierarchy"}\NormalTok{ =}\StringTok{ }\DecValTok{3}\NormalTok{),}
  \DataTypeTok{width =} \DecValTok{25}\NormalTok{,}
  \DataTypeTok{height =} \DecValTok{20}
\NormalTok{)}
\end{Highlighting}
\end{Shaded}

\texttt{rsvg::rsvg\_png} could be applied to convert .svg to other png
file.

\begin{Shaded}
\begin{Highlighting}[]
\NormalTok{from.svg \textless{}{-}}\StringTok{ }\KeywordTok{paste0}\NormalTok{(}
\NormalTok{  .MCn.output, }\StringTok{"/"}\NormalTok{, .MCn.results,}
  \StringTok{"/parent\_nebula/parent\_nebula.svg"}
\NormalTok{)}
\NormalTok{to.png \textless{}{-}}\StringTok{ }\KeywordTok{paste0}\NormalTok{(}
\NormalTok{  .MCn.output, }\StringTok{"/"}\NormalTok{, .MCn.results,}
  \StringTok{"/parent\_nebula/parent\_nebula.png"}
\NormalTok{)}
\NormalTok{rsvg}\OperatorTok{::}\KeywordTok{rsvg\_png}\NormalTok{(from.svg, to.png)}
\end{Highlighting}
\end{Shaded}

\hypertarget{visualization-of-child-nebula}{%
\subparagraph{Visualization of
child-nebula}\label{visualization-of-child-nebula}}

This function draw child-nebula in grid panel. The overview of
child-nebula shows the abundant classes of this LC-MS/MS dataset.

\begin{Shaded}
\begin{Highlighting}[]
\NormalTok{MCnebula}\OperatorTok{::}\KeywordTok{visualize\_child\_nebulae}\NormalTok{()}
\end{Highlighting}
\end{Shaded}

All output files:

\begin{Shaded}
\begin{Highlighting}[]
\KeywordTok{list.files}\NormalTok{(}\KeywordTok{paste0}\NormalTok{(.MCn.output, }\StringTok{"/"}\NormalTok{, .MCn.results), }\DataTypeTok{recursive =}\NormalTok{ T)}
\CommentTok{\#\textgreater{}  [1] "child\_nebula/Alkanolamines.graphml"                   }
\CommentTok{\#\textgreater{}  [2] "child\_nebula/Aralkylamines.graphml"                   }
\CommentTok{\#\textgreater{}  [3] "child\_nebula/Benzenediols.graphml"                    }
\CommentTok{\#\textgreater{}  [4] "child\_nebula/Benzoic acid esters.graphml"             }
\CommentTok{\#\textgreater{}  [5] "child\_nebula/Benzoyl derivatives.graphml"             }
\CommentTok{\#\textgreater{}  [6] "child\_nebula/Catechols.graphml"                       }
\CommentTok{\#\textgreater{}  [7] "child\_nebula/Coumarins and derivatives.graphml"       }
\CommentTok{\#\textgreater{}  [8] "child\_nebula/Dialkyl ethers.graphml"                  }
\CommentTok{\#\textgreater{}  [9] "child\_nebula/Disaccharides.graphml"                   }
\CommentTok{\#\textgreater{} [10] "child\_nebula/Diterpenoids.graphml"                    }
\CommentTok{\#\textgreater{} [11] "child\_nebula/Fatty acids and conjugates.graphml"      }
\CommentTok{\#\textgreater{} [12] "child\_nebula/Flavans.graphml"                         }
\CommentTok{\#\textgreater{} [13] "child\_nebula/Flavonoid glycosides.graphml"            }
\CommentTok{\#\textgreater{} [14] "child\_nebula/Flavonoids.graphml"                      }
\CommentTok{\#\textgreater{} [15] "child\_nebula/Furans.graphml"                          }
\CommentTok{\#\textgreater{} [16] "child\_nebula/Gamma butyrolactones.graphml"            }
\CommentTok{\#\textgreater{} [17] "child\_nebula/Hydroxy acids and derivatives.graphml"   }
\CommentTok{\#\textgreater{} [18] "child\_nebula/Indoles and derivatives.graphml"         }
\CommentTok{\#\textgreater{} [19] "child\_nebula/Lactams.graphml"                         }
\CommentTok{\#\textgreater{} [20] "child\_nebula/Methoxybenzenes.graphml"                 }
\CommentTok{\#\textgreater{} [21] "child\_nebula/Methoxyphenols.graphml"                  }
\CommentTok{\#\textgreater{} [22] "child\_nebula/Monoterpenoids.graphml"                  }
\CommentTok{\#\textgreater{} [23] "child\_nebula/O{-}methylated flavonoids.graphml"         }
\CommentTok{\#\textgreater{} [24] "child\_nebula/Organosulfur compounds.graphml"          }
\CommentTok{\#\textgreater{} [25] "child\_nebula/Oxosteroids.graphml"                     }
\CommentTok{\#\textgreater{} [26] "child\_nebula/Pregnane steroids.graphml"               }
\CommentTok{\#\textgreater{} [27] "child\_nebula/Primary amines.graphml"                  }
\CommentTok{\#\textgreater{} [28] "child\_nebula/Pyridines and derivatives.graphml"       }
\CommentTok{\#\textgreater{} [29] "child\_nebula/Pyrrolidines.graphml"                    }
\CommentTok{\#\textgreater{} [30] "child\_nebula/Sesquiterpenoids.graphml"                }
\CommentTok{\#\textgreater{} [31] "child\_nebula/Steroids and steroid derivatives.graphml"}
\CommentTok{\#\textgreater{} [32] "child\_nebula/Styrenes.graphml"                        }
\CommentTok{\#\textgreater{} [33] "child\_nebula/Terpene glycosides.graphml"              }
\CommentTok{\#\textgreater{} [34] "child\_nebula/Terpene lactones.graphml"                }
\CommentTok{\#\textgreater{} [35] "child\_nebula/Tertiary amines.graphml"                 }
\CommentTok{\#\textgreater{} [36] "child\_nebula/Tetrahydrofurans.graphml"                }
\CommentTok{\#\textgreater{} [37] "child\_nebula/Triterpenoids.graphml"                   }
\CommentTok{\#\textgreater{} [38] "child\_nebula/Vinylogous esters.graphml"               }
\CommentTok{\#\textgreater{} [39] "method\_pick\_formula\_excellent.structure.tsv"          }
\CommentTok{\#\textgreater{} [40] "method\_pick\_formula\_excellent.tsv"                    }
\CommentTok{\#\textgreater{} [41] "nebula\_index.tsv"                                     }
\CommentTok{\#\textgreater{} [42] "parent\_nebula/parent\_nebula\_edges.tsv"                }
\CommentTok{\#\textgreater{} [43] "parent\_nebula/parent\_nebula\_nodes.tsv"                }
\CommentTok{\#\textgreater{} [44] "parent\_nebula/parent\_nebula.graphml"}
\end{Highlighting}
\end{Shaded}




\end{document}
